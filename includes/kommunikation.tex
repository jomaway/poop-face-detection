\section{Kommunikation}
\label{sec:com}

Dieses Kapitel geht im näheren auf die Kommunikation zwischen dem Sensor und dem Evaluation System ein. 


\subsection{Anforderungen}

Die Kommunikation zwischen dem Sensor und dem Evaluation System ist ein wesentlicher Bestandteil des gesamten Projektes. Im nächsten Abschnitt wird näher auf die Anforderungen dieser eingegangen.

\begin{description}
  \item[Mobilität:]
  Das System soll so ausgelegt sein, dass die Eltern das Kind auch einfach aus dem Bett heben können ohne vorher Kabelverbindungen trennen zu müssen oder das Gefahr besteht das System dabei zu beschädigen. Daher soll eine kabellose Verbindung verwendet werden.
  
  \item[Einfachheit:]
  Für die Entwicklung des Prototyps sollte die Kommunikation nicht zu Komplex gestalltet werden um die vorhandenen personellen Ressourcen nicht zu übersteigen. 
  
  \item[Energie sparend:]
  Da der Sensor langfristig über einen Akku betrieben werden soll, ist es nötig auf eine Energie sparende Technology zu setzen, damit der Akku nicht jeden Tag geladen werden muss.
  
  \item[Reichweite:]
  Die maximale Reichweite muss ca 2-3 Meter betragen, wenn man den typischen Anwendungsfall betrachtet, dass ein Kind in seinem Bett liegt und das Kamerasystem etwas darüber angebracht wird.
  
  \item[Erweiterbarkeit:]
  Die wahl der Kommunikationstechnology sollte auch im Hinblick auf eine einfache Erweiterbarkeit in der Zukunft gewählt werden. Dabei betrachten wir folgende Ideen:
  \begin{itemize}
    \item Direkte Verbindung von einem Smartphone zum Sensor
    \item Bau einer eigenen Platine zum Sensor 
  \end{itemize}
   
\end{description}

\subsection{Umsetzung}
Aufgrund der oben gennanten Anforderungen setzten wir auf die Technology Bluetooth Low Energy (im folgenden mit BLE abgekürzt). Der Raspberry Pi V3 auf dem unser Evaluation System aufbaut bringt diese Schnittstelle von Haus aus mit und am Sensor kommt für unseren Prototyp ein HM10-Modul zum Einsatz. BLE ist ein auf energiesparende Systeme ausgelegtes, drahtloses Netwerkprotokol. Mit einer Reichweite von 10 Meter \cite{ble_spec} bietet es genügend Spielraum für unseren Anwendungsbereich. Im folgenden wird zuerst kurz auf die Grundlagen von Bluetooth Low Energy eingegangen und anschließend die Einrichtung und damit verbundenen Probleme diskutiert. 

\subsubsection{BLE Grundlagen}
\textcolor{red}{TODO: GRUNDLAGEN KAPITEL}

\subsubsection{Unser Übertragungsprotokoll}
\label{subsubsec:unser_protokoll}
- Übertragung des Zustandes DRY oder WET an das Eval System.
- Im Moment keine übertragung in gegengesetzte richtung, aber evtl in zukunft vorstellbar ,
um konfigurationen durchzuführen oder test durchzuführen. 

- sendet in 10 sekunden abständen -> besseres debugging
 -> mehrmals WET senden falls eins verloren geht oder nicht erkannt wird
- sendet von nur noch WET zuständen um weiter energie zu sparen.

- verbesserungs für v2 -> nur noch senden wenn WET. 
 bzw in abständen von 5-10 min mal DRY um sicherzustellen das verbindung noch besteht. 
 (evlt kann man auch so erkennen das es besteht)


\subsubsection{Einrichtung Sensorseite}
Im aktuellen Prototyp kommt ein HM10-BLE-Modul zum einsatz. Über eine Serielle Schnittstelle kann das Modul konfiguriert und genutzt werden. Dies bietet einige Vorteile. Das Modul kommt in der Regel als Slave vorkonfiguriert und kann als solches direkt verwendet werden. Dabei werden alle Nachrichten die man an die Serielle Schnittstelle sendet auf den BLE Service mit der ID "FFEE"\textcolor{red}{(TODO: überprüfen)} gemapped und über diese versendet. In der Gegenrichtung werden äquivalent alle empfangenen Nachrichten über die Serielle Schnittstelle ausgegeben. Dies ermöglicht eine einfache Nutzung der Bluetooth Low Energy Technology für kleine Projekte, da man sich nicht tiefgreifend mit dem Protokoll auseinander setzen muss. Dabei wird aber auch der Nachteil sichtbar. So ist es nicht Möglich eigene BLE Services zu definieren. Dies ist jedoch für unseren Anwendungsbereich nicht von bedeutung. 

Die Konfiguration des HM10 Moduls erfolgt über sogenannte AT-Kommandos \footnote{\textcolor{red}{TODO:} }, welche ursprünglich zur Konfiguration von Modems entwickelt wurde. Eine Liste mit allen verfügbaren Kommandos ist unter \textcolor{red}{TODO } zu finden. Beim Kauf eines solchen Modules sollte man allerdings darauf zu achten, dass man kein Plagiat bekommt, da diese in ihrer Funktionalität oftmals eingeschränkt sind \textcolor{red}{(TODO: ref)}. Eine Möglichkeit herauszufinden, ob man ein solches bekommen hat bietet ein Projekt von \textcolor{red}{(TODO: name)} welches unter \textcolor{red}{(TODO: githublink einfügen)} zu finden ist.

\paragraph{Impementierung}
Um die Bedienung für das Sensorteam so einfach wie möglich zu gestallten, wurde eine C Bibliothek entwickelt, welche die Kommunikation mit dem Modul regelt. Diese stellte zwei funktionen zu verfügung \textcolor{red}{(TODO: listing)} send() % setMode(). Dabei wird die send() funktion, durch einen Interrupt Timer, zyklisch alle 10 sekunden wie im abschnitt \ref{subsubsec:unser_protokoll} beschrieben aufgerufen. Detektiert der Sensor eine nasse Windel muss nur der mode, mittels der funktion setMode() auf "WET" umgeschalten werden.

Der komplette Code ist auf Github \textcolor{red}{(TODO link einfügen )}zu finden.

Wird eine Platine für den Sensor entwickelt, besteht die Möglichkeit, weiterhin ein HM10-Modul and den Microcontroller anzuschließen oder einen Chip mit integriertem BLE zu verwenden und dort den gleichen Service zu implementieren. In beiden Fällen muss an der Implementierung vom Evaluation System, welche im Kapitel \ref{subsec:ble_raspi} beschrieben wird, nichts geändert werden.

\subsubsection{Einrichtung Raspberry Pi}
Wie weiter oben bereits erwähnt, besitzt der Raspberry Pi in der Version 3 bereits einen BLE Chip. Dadurch wird keine weiter Hardware benötigt. Auf dem Raspi läuft das Linux Betriebsystem Raspbian, welches den Bluetooth Stack bluez mitbringt. Da die implementierung der Gatt Profile auf dem Evaluation System einige Probleme mit sich bringt, wird im nächsten Abschnitt zunächst auf diese eingegangen.

\paragraph{Probleme}. \\ 
Ein wesentlicher Punkt ist die Verbreitung von Bluetooth Low Energy im allgemeinen. Während BLE bereits in vielen Projekten im Sensor-, Embedded- und Mobilbereich eingesetzt wird, ist der Einsatz auf einem klassischen Betriebsystem weniger von bedeutung \textcolor{red}{(TODO: quelle raussuchen?)}. Dadurch sind kaum brauchbare Dokumentationen und Tutorials zu diesem Thema zu finden. Zwar liefert der BLE Stack bluez seit einiger Zeit eine Implementierung für BLE anwendungen mit, jedoch gilt auch hier, wie bei vielen Open Source Projekten, der Quellcode ist die Dokumentation. Dadurch ist die Hürde für Einsteiger sehr hoch. Ein im Rahmen dieses Projekt entstandenes Papers "Bluetooth Low Energy Programming on Linux"  \textcolor{red}{TODO: ref einfügen}, welches im  \textcolor{red}{TODO:ref  Anhang X} beiliegt, beschreibt den Einstieg in die BLE programmierung unter linux, daher wird hier nicht näher darauf eingegangen. 

\paragraph{Implementierung}. \\

- Dateistruktur.
- Gattlib
- Ablauf : Detektieren des Signals -> Erkennen ob neues Signal oder wieder holung des alten
  -> triggerfile erstellen.
  


\subparagraph{Zusammenspiel mit Aufnahemsystem}
