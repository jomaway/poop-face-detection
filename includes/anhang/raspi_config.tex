\section{Einrichtung des Raspberry Pi}
\label{sec:Einrichtung_Raspi}

Hier wird kurz die Einrichtung des Betriebsystems Raspbian auf dem Raspberry Pi erläutert.

\subsection{Raspian}

Als Betriebssystem wurde sich für das auf debian basierende Raspbian entschieden. Im folgenden wird kurz geschildert, welche Schritte nötig sind, um dies auf dem Raspberry Pi einzurichten.

\begin{enumerate}
  \item Download des Image unter raspberrypi.org (Version: Rasbian Jessi Pixel, da diese eine gui mitbringt, welche zum Prototyping von Vorteil ist.)
  \item SD-Karte mittels lsdisk ermitteln (Müsste ähnlich wie /dev/mmcblk0 aussehen)
  \item Image mittels dem Tool dd auf eine SD-Karte übertragen (Beispiel: dd if=/path/rasbian-image/2017-04-10-raspbian-jessie.img of=/dev/mmcblk0)
  \item Damit auch genügend Speicherplatz für die Videos vorhanden ist muss die Größe der Partition erweitert werden. \\
  \begin {enumerate}
    \item fdisk /dev/mmcblk0
    \item Mittels p alle Partitionen anzeigen lassen. \\ Man sieht hier 2 Partitionen eine kleine boot Partition und eine etwas größere Linux Partition. Von dieser 2ten   Partition müssen wir uns den StartSector notieren. 
    \item Anschließend kann mittels d die zweite Partition gelöscht werden. 
    \item Mittels n kann diese wieder erstellt werden. Hier bei startVektor das notierte ergebnis eingeben, sonst stimmt die partitionszuordnung nicht mehr. 
    \item Als endvektor den standart wert übernehmen.
    \item Danach mit w die änderungen auf die sdkarte schreiben.
    \item Nach einem neustart muss man noch das Dateisystem mittels sudo resize2fs /dev/mmcblk0p2 anpassen.
    \item Mit df -h sieht man nun das die neue größe der partition.
    \item Link: http://www.fabiandeitelhoff.de/2014/07/raspberry-pi-speicherplatz-der-sd-karte-ausnutzen/
  \end{enumerate}
\end{enumerate}



\subsection{Grundkonfiguration des Raspberry Pi}

Für eine bessere Benutzbarkeit des Systems wurden einige Grundkonfigurationen vorgenommen. Generelle Informationen dazu sind unter \url{https://www.elektronik-kompendium.de/sites/raspberry-pi/1906291.htm} zu finden.

\paragraph{Hostname} \hspace{0pt} \\
Der Hostname des Raspi wird auf poopface\_eval\_sys geändert
\paragraph{Password}
Das Passwort nach bedarf ändern %never save a Password in a git repo
\paragraph{Interfaces} \hspace{0pt} \\
Enable the SSH and the Camera interface
\begin{itemize}
  \item ssh
  \item camera
\end{itemize}

\paragraph{Boot options} \hspace{0pt} \\
boot to cli with autologin

\paragraph{WLAN} \hspace{0pt}\\
Um WLAN über eine Konsole einzurichten, muss man sich mittels su als root einloggen und folgende Schritte durchführen

\begin{itemize}
  \item iwlist wlan0 scan // scan all wifis to show ESSID:
  \item wpa\_cli reconfigure // if not recorgnized automatically
  \item ifconfig wlan0 up // if wlan is not up
  \item ifconfig wlan0 // to get the address from which you can connect to via SSH.
  \item delete \#psk="PASSWORD" line from /etc/wpa\_supplicant/wpa\_supplicant.conf
\end{itemize}

Weitere Informationen unter \url{ https://www.raspberrypi.org/documentation/configuration/wireless/wireless-cli.md}.

\paragraph{Software and Updates} \hspace{0pt} \\
Check for updates via sudo apt update \&\& sudo apt upgrade

Installiere falls noch nicht vorhanden:
- python3
- python3-pip 
\begin{itemize}
  \item python3
  \item python3-pip
\end{itemize}


